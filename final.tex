% Options for packages loaded elsewhere
\PassOptionsToPackage{unicode}{hyperref}
\PassOptionsToPackage{hyphens}{url}
%
\documentclass[
]{article}
\usepackage{lmodern}
\usepackage{amssymb,amsmath}
\usepackage{ifxetex,ifluatex}
\ifnum 0\ifxetex 1\fi\ifluatex 1\fi=0 % if pdftex
  \usepackage[T1]{fontenc}
  \usepackage[utf8]{inputenc}
  \usepackage{textcomp} % provide euro and other symbols
\else % if luatex or xetex
  \usepackage{unicode-math}
  \defaultfontfeatures{Scale=MatchLowercase}
  \defaultfontfeatures[\rmfamily]{Ligatures=TeX,Scale=1}
\fi
% Use upquote if available, for straight quotes in verbatim environments
\IfFileExists{upquote.sty}{\usepackage{upquote}}{}
\IfFileExists{microtype.sty}{% use microtype if available
  \usepackage[]{microtype}
  \UseMicrotypeSet[protrusion]{basicmath} % disable protrusion for tt fonts
}{}
\makeatletter
\@ifundefined{KOMAClassName}{% if non-KOMA class
  \IfFileExists{parskip.sty}{%
    \usepackage{parskip}
  }{% else
    \setlength{\parindent}{0pt}
    \setlength{\parskip}{6pt plus 2pt minus 1pt}}
}{% if KOMA class
  \KOMAoptions{parskip=half}}
\makeatother
\usepackage{xcolor}
\IfFileExists{xurl.sty}{\usepackage{xurl}}{} % add URL line breaks if available
\IfFileExists{bookmark.sty}{\usepackage{bookmark}}{\usepackage{hyperref}}
\hypersetup{
  hidelinks,
  pdfcreator={LaTeX via pandoc}}
\urlstyle{same} % disable monospaced font for URLs
\setlength{\emergencystretch}{3em} % prevent overfull lines
\providecommand{\tightlist}{%
  \setlength{\itemsep}{0pt}\setlength{\parskip}{0pt}}
\setcounter{secnumdepth}{-\maxdimen} % remove section numbering

\date{}

\begin{document}

\hypertarget{header-n0}{%
\subsection{布朗运动与 DNA 随机行走}\label{header-n0}}

\hypertarget{header-n2}{%
\paragraph{摘要}\label{header-n2}}

参考文献《布朗运动100年》,本文首先简要介绍一下布朗运动与随机行走在数学上的联系。接着采用数值模拟的方法来详细说明这一点。然后就文献中提到的
DNA Walker 现象进行数值计算。

\hypertarget{header-n4}{%
\subsubsection{扩散方程与随机行走}\label{header-n4}}

以下内容来自参考文献{[}{]},这里只是做一下简要回顾。著名的扩散方程具有以下形式,

\[\frac{\partial \rho}{\partial t}=D \frac{\partial^{2} \rho}{\partial x^{2}}\]

假定在 \(t =0\) 时刻粒子位于 \(x =0\) 处,即
\(\rho(x , 0) = \delta ( x )\),扩散方程的解是:

\[\rho(x, t)=\frac{1}{\sqrt{4 \pi D t}} e^{\frac{x^{2}}{4 D t}},\]

即粒子的密度遵从高斯分布。对于固定的时刻 \(t\) ,有,

\[\langle x\rangle= 0, \quad\left\langle x^{2}\right\rangle= 2 D t,\]

可以验证
\(\displaystyle{\int_{-\infty}^{\infty}\rho(x,t) \mathrm{d}\,t=1}\)。这样就得到了扩散长度公式,

\[\sqrt{\left\langle x^{2}\right\rangle}=\sqrt{2 D t},\]

这里出现了著名的爱因斯坦的 \(\frac{1} {2} \) 指数。

\end{document}
