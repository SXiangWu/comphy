%!TeX program = xelatex
\documentclass[
reprint,
amsmath,
amssymb,
aps,
%pra,
prl
%rmp,
%prstab,
%prstper,
%floatfix,
]{revtex4-1}
\usepackage{ctex}
\usepackage{graphicx}% Include figure files
\usepackage{dcolumn}% Align table columns on decimal point
\usepackage{bm}% bold math
\usepackage{subfigure}
% \usepackage{hyperref}% add hypertext capabilities
\newcommand{\RomanNumeralCaps}[1]{\MakeUppercase{\romannumeral #1}}
\makeatother

\begin{document}
\title{布朗运动与DNA 随机行走}
  \author{SA18002024  吴双祥}
  \affiliation{中国科学技术大学物理学院}%
  \date{\today}

  \begin{abstract}
    参考文献“布朗运动100年”,本文首先简要介绍一下布朗运动与随机行走在数学上的联系。
    接着采用数值模拟的方法来详细说明这一点。
    然后就文献中提到的 DNA Walker 现象进行
  \end{abstract}

  \maketitle

  \section{扩散方程与随机行走}%
  \label{sec:bu_lang_yun_dong_yu_sui_ji_xing_zou_}
  著名的扩散方程具有以下形式,
  \begin{equation}
     \frac{\partial \rho}{\partial t} = \frac{\partial^2}{denom}  
  \end{equation}
 
 \end{document}
